\documentclass[12pt]{scrartcl}

% Font and input encoding
\usepackage{fontspec}
\defaultfontfeatures{Ligatures=TeX,Renderer=Basic}
\usepackage{polyglossia}
\setmainlanguage{russian}
\setotherlanguage{english}

% Explicitly setting fonts that support Cyrillic
\setmainfont{CMU Serif}
\setsansfont{CMU Sans Serif}
\setmonofont{CMU Typewriter Text}

% For math symbols
% Include AMS packages for mathematical symbols and fonts
\usepackage{amsmath}
\usepackage{amssymb}

% For code listings
\usepackage{listings}

% For epigraph
\usepackage{epigraph}

% For tables
\usepackage{tabularx}
\usepackage{placeins}

% For drawing
\usepackage{pgfplots}
\pgfplotsset{compat=1.18}
\usepackage{pgfplotstable}
\usetikzlibrary{pgfplots.statistics}
\usetikzlibrary{pgfplots.groupplots} % Load the groupplots library

% For files reading
\usepackage{filecontents}


% Bibliography management (optional in presentation, often omitted)
\usepackage[backend=biber]{biblatex}
\addbibresource{references.bib} % Specify your bibliography file here

% For clickable links (often less necessary in presentations)
\usepackage{hyperref}

% Settings for links (optional in presentations)
\hypersetup{
	colorlinks=true,
	linkcolor=red,
	filecolor=magenta,
	urlcolor=cyan,
}

% Document metadata
\title{Кризисные явления и выбор пути независимой оценки в России в 2024 году}
\subtitle{итоги Круглого Стола Союза СРОО и НМСО, прошедшего 04 июля 2024}

% Define authors
\author{К.\,А.~Мурашев\thanks{email: kirill.murashev@gmail.com, \href{https://t.me/AIinValuation}{Telegram}}}

\date{\today}

\begin{document}

\maketitle

\epigraph{\emph{Реальной проблемой человечества является то, что мы являемся обладателями эмоций эпохи Палеолита, средневековых институтов и богоподобных технологий. Это ужасающе опасно и в данный момент ведёт нас к точке общего кризиса.}

    Edward O. Wilson.
}

\begin{abstract}
04 июля 2024 состоялся Круглый стол Комитета по Научным и Методологическим Вопросам в Оценочной Деятельности Союза Саморегулируемых Организаций Оценщиков (Национальное Объединение Оценщиков) и Санкт-Петербургского Научно-Методического Совета по Оценке на тему «Кризисные явления в методологии оценки недвижимости. Источники информации в практике российской оценки. Какая информация является недостоверной в оценке рыночной стоимости». Запись и презентации выступлений размещены на сайте СРОО СПО по \href{https://cpa-russia.org/news-spo/2750/}{ссылке}~\cite{SROO202407} . В данном материале представлен один из взглядов на причины проблем и предлагается набор действий для разворота сложившегося тренда на исчезновение института независимой оценки в России. Ряд идей и предложений, высказанных в данной работе, могут показаться спорными либо дискуссионными. Содержание статьи является отражением мнения её автора, сформированного по итогам состоявшегося Круглого Стола и может не отражать мнения других авторов докладов и иных участников.

\bigskip
\textenglish{On July 04, 2024 a round-table of the Committee on Scientific and Methodological Issues in Appraisal Activities of the Union of Self-Regulatory Organizations of Appraisers (National Association of Appraisers) and the St. Petersburg Scientific and Methodological Council on Appraisal was held on the topic `''Crisis phenomena in real estate appraisal methodology. Sources of information in Russian appraisal practice. What information is unreliable in the assessment of market value''. The recordings and presentations of the speeches are available on the SROA UAP website at \href{https://cpa-russia.org/news-spo/2750/}{link}~\cite{SROO202407}. This paper presents one of the views on the causes of the problems and proposes a set of actions to reverse the current trend towards the disappearance of the institution of independent appraisal in Russia. A number of ideas and suggestions made in this paper may seem controversial or debatable. The content of the article is a reflection of the opinion of its author formed on the basis of the results of the Round Table and may not reflect the opinion of other authors of the reports and other participants.}

\end{abstract}

\section{Введение}\label{sec:Introduction}
Основной темой прошедшего Круглого стола являлся вопрос кризисных явлений, наблюдающихся в отрасли независимой оценки примерно с 2013 года. Также были рассмотрены некоторые аспекты некорректной практики применения данных и информации со стороны части независимых оценщиков. Были высказаны предположения о возможном наличии связи между этими явлениями. Поскольку вопрос носит достаточно общий характер, представляется целесообразным рассмотреть основные этапы развития оценочной деятельности в России в контексте событий в экономике и институциональной среде. В таблице~\ref{tab:rus_history} на странице~\pageref{tab:rus_history} приводится обзор основных этапов развития оценочной деятельности в России в контексте фаз развития экономики. С точки зрения ``технического анализа'', можно осторожно сделать вывод, что фазы развития экономики как правило имеют длину около 5 лет, а порядковые номера годов перехода от одной фазе к другой оканчиваются на 3 либо 8. Данное наблюдение носит приблизительный характер и отражает лишь упрощённое понимание истории экономики России. Однако, в первом приближении, такое выделение фаз представляется достаточно удобным для анализа. Из фактов, приведённых в таблице, следует, что в целом развитие оценочной деятельности и её состояние достаточно точно следуют трендам в экономике в целом. При этом, состояние отрасли является более волатильным: в период роста экономики, отрасль растёт быстрее экономики в целом, в период спада --- испытывает более серьёзные трудности, чем экономика в целом. Данная закономерность была нарушена в 2023 году, когда на фоне роста экономики, а также реальных доходов граждан, отрасль независимой оценки продолжила падение. Дальнейший отказ кредитных организаций от сотрудничества с внешними оценщиками, а также потенциальное исключение независимых оценщиков из процедуры оспаривания кадастровой стоимости равно как и проведения любых видов судебных экспертиз создали реальную угрозу практически полного исчезновения профессии.

С учётом факта, что определённые кризисные явления в области независимой оценки наблюдаются не только в России, представляется целесообразным рассмотреть достаточно широкий круг вопросов, связанных с местом оценщика в современной экономике. В данном материале будет предпринята попытка рассмотреть макроэкономические, институциональные и технологические причины кризисных явлений, обозначенных организаторами и участниками Круглого Стола. Такой взгляд возможно позволит отойти от узких проблем конкретных методов и техник, использование которых усугубляет кризисные явления в отрасли.

\begin{table}[h]
    \scriptsize
	\caption{История развития оценки в контексте фаз экономики России}
	\label{tab:rus_history}
	\centering
    \begin{tabularx}{\textwidth}{>{\hsize=0.23\hsize}X >{\hsize=0.9\hsize}X X}
		\hline
		Период & Экономика & Оценка\\
		\hline \hline
		1988--1993 & Формирование квазирыночных практик, первоначальное накопление капитала, бартерные отношения, формирование фактических, но не юридических прав собственности на некоторые активы. & Отсутствие потребности в данном виде деятельности в силу отсутствия единообразия практик и правовой базы.\\
		\hline
		1993--1998 & Создание правовой базы института частной собственности, формирование класса собственников, выбор модели суверенной приватизации без допуска иностранных инвесторов. Выход экономики на траекторию роста. & Доинституциональный период: появление практики оценки стоимости как консалтинга вне рамок правовых процедур, первичное обучение оценщиков на основе практики оценщиков США, а также Всемирного Банка. Появление первого оценщика в России: Е.\,И.~Неймана. Отсутствие различия между внутренней и внешней (независимой) оценкой. \\
		\hline
		1998--2003 & Оздоровление экономики и бюджетной системы в т.\,ч. через механизмы девальвации и дефолта. Развитие производства и внутреннего спроса. Устранение фактора силового давления на бизнес со стороны несистемных агентов. & Возникновение института оценки в рамках правового поля, активное применение заимствованных практик и формирование собственной научной и практической базы. Создание в Санкт-Петербурге национальной научной школы оценки: Е.\,С.~Озеров, Д.\,Д.~Кузнецов, В.\,Н.~Мягков и др. Разработка собственных математических методов оценки в т.\,ч. силами таких известных специалистов как Д.\,Д.~Кузнецов, С.\,В.~Грибовский, В.\,Г. Мисовец\,\textdagger и др.\\
		\hline
		2003--2008 & Сырьевой "суперцикл", исключительно высокий темп роста экономики и благосостояния граждан. Высокая капитализация российских компаний, позитивный образ российской экономики, активность иностранных инвесторов, преодоление большей части проблем переходного периода. & Расцвет отрасли: высокий статус и доходы независимых оценщиков, доверие к институту,в т.ч. переход на модель физических как субъектов оценочной деятельности. Активное развитие методологии силами широкого круга специалистов. Сочетание применения иностранных и собственных методик. Возникновение института внутренней оценки, его взаимопроникновение с институтом внешней оценки.\\
		\hline
		2008-2013 & Мировой экономический кризис и его преодоление. Умеренный темп роста экономики. Достижение высшего уровня развития экономики и благосостояния граждан за всю историю. Начало процесса цифровизации экономики. & Максимальный уровень развития отрасли. Разнообразие методологии и её дальнейшее развитие. Высокое доверие к отрасли, в т\,ч. выполнение кадастровой оценки силами частных компаний. Наивысший уровень доходов оценочной бизнеса и оценщиков. Продолжение развития института внутренней оценки в общем русле с внешней.\\
		\hline
		2013--2018 & Перелом тренда, остановка роста экономики и её дальнейшее падение. Падение доходов граждан. Усиление налоговой нагрузки, опережающий рост государственных доходов и расходов относительно экономики. Развитие цифровых практик в экономике. & Перелом тренда, падение доходов. Резкое упрощение методологии, применяемой внешними оценщиками. Начало утраты профессией оценщика фактического статуса профессиональной деятельности. Усиление института ``оценка исключительно как бизнес''. Разделение путей внешней и внутренней оценки. Сохранение профессионального уровня и его дальнейшее развитие в среде внутренних оценщиков. Недовольство заказчиков и регулятора деятельностью оценщиков. Изъятие части направлений оценки из под действия 135-ФЗ, передача их иным специалистам. Попытка исправления ситуация путём изменения правовой основы деятельности посредством введения квалификационного экзамена и разделения направлений деятельности.\\
		\hline
		2018--2023 & Смешанная картина: рост в государственном секторе экономики, разнонаправленное движение в частном, продолжение усиления налоговой нагрузки и администрирования, а также падения доходов граждан. Продолжение развития цифровых практик. Формирование в некоторых областях лучших в мире практик цифровизации. Активная роль государства по всех сферах экономики. Мягкая денежно-кредитная политика. & Дальнейшее снижение качества работы независимых оценщиков, открытое недовольство их деятельностью, продолжение попыток изъятия тех или иных видов оценки у независимых оценщиков. Окончательное разделение внешней и внутренней оценки. Последовательный курс на упрощение внешней оценки. Развитие внутренней оценки, в т.\,ч. активное применение современных цифровых методов оценки. \\
		\hline
		2023--? & Перестройка экономики в условиях новой реальности. Сочетание моделей военного кейнсианства и общего стимулирования потребительских расходов. Рост во многих секторах, угроза ``перегрева'' экономики, попытки обуздания роста денежной массы. Рост доходов граждан. Дальнейшее усиления налогового пресса на бизнес и граждан. Разворот в денежно-кредитной политике. Один из самых высоких уровней цифровизации экономики в мире. & Сохранение всех негативных трендов в независимой оценке. Развитие тренда на отказ от услуг внешних оценщиков. Очередной этап усиления регуляции и требований к квалификации без видимого положительного результата. Разрыв связи между внешней и внутренней оценкой. Активное развитие методов оценки в среде внутренних оценщиков. Реальная угроза исчезновения института независимой оценки в России.\\
		\hline
	\end{tabularx}
    \normalsize
\end{table}

\section{Оценщик, его статусы и функции}\label{sec:appraiser}
Для того чтобы понять . В первую очередь следует обозначить очевидный факт: ценообразование почти на все товары и услуги осуществляется без участия оценщиков либо иных консультантов. Рыночные агенты успешно справляются с задачами определения цен самостоятельно в рамках своей обычной деятельности. Таким образом, можно достаточно уверенно сказать, что основной целью работы оценщика является не заполнение пробелов в знаниях рыночных агентов о стоимости имеющих для них значение объектов, прав и обязательств.

\subsection{Виды}
Следует отметить, что существует два вида оценщиков: внутренние и независимые (внешние). При этом, в российском правовом поле отсутствует такое понятие как ``внутренний оценщик''. Предполагается, что оценщик может осуществлять свою деятельность только независимо. Фактическое положение дел совершенно обратное. Формально независимые оценщики часто не могут рели.

\subsubsection{История вопроса}
Вместе с тем, институт внутренней оценки (в т.\,ч. оценки для целей налогообложения силами уполномоченных со стороны территориальной либо полисной администрации) существует с момента начала письменной истории человечества. Упоминания об оценке как систематизированной процедуре встречаются в самых первых известных материальных источниках. В определённом смысле можно сказать, что возникновение института внутренней оценки стало одним из индикаторов перехода из доисторического Неолита в исторический период. При этом, оценка как таковая сохраняла своё значение во все историко-экономические эпохи. Следует отметить, что на протяжении очень долгого периода исторической жизни человечества, основным, а часто и практически единственным ценным активом, была земля для сельскохозяйственного производства. При этом, также следует знать, что также на протяжении большей части исторического периода, она не была оборотоспособна в современном понимании сделки купли-продажи. В архаичных обществах владение землёй часто носило общинный либо родовой характер. Земля и проживающие на ней люди, обрабатывающие её, были неразделимы и вопрос о продаже земли не мог стоять в принципе. Определённый рынок земли существовал в римский период, хотя и там земля чаще выдавалась за определённые заслуги, чем приобреталась за деньги. В Средние Века, в рамках феодальных отношений, земля практически не имела оборотоспособность и была неразрывно связана с личностями конкретных владельцев, а также системой старшинства, в т.\,ч. вассальных отношений. При этом, сохранялась система общинного либо иного совместного владения землёй у самих крестьян, составлявших в те времена свыше 90\,\% населения. Возможность продажи земли как любого другого актива постепенно начала появляться только с развитием капитализма. При этом, на ряде территорий Восточной Европы общинный характер владения землёй мог сохраняться практически до ХХ века. Тем не менее, данное обстоятельство не мешало как-либо проведению оценки стоимости земли, являющейся первым известным видом оценки. Её оценка проводилась на основании её продуктивности, как правило годовой, т.\,е. с помощью методов доходного подхода. Наиболее вероятная последовательность возникновения подходов выглядит следующим образом:
\begin{enumerate}
    \item Доходный подход;
    \item Сравнительный подход;
    \item Затратный подход.
\end{enumerate}
В целом, можно сказать, что оценка и её методы намного устойчивее любых конкретных форм организации жизни людей.

\paragraph{\href{https://en.wikipedia.org/wiki/Hydraulic_empire}{Ирригационные протогосударства}}
В Древнем Египте оценка была неотъемлемой частью управления государством, особенно для целей налогообложения. Египтяне использовали систему оценки земли для определения размера налогов, которые должны были платить крестьяне, исходя из размера и продуктивности их земель~\cite{hydraulic_empire}, \cite{Wittfogel1981}, \cite{Trigger1983}. Интересным представляется, что и расшифровка древнеегипетской письменности стала возможной благодаря изучению \href{https://en.wikipedia.org/wiki/Rosetta_Stone}{Розеттского камня}, содержащего сведения в т.\,ч. о вопросах имущественных налогов, определяемых на основе оценки.

В Древней Месопотамии, в частности в Ассирии и Вавилоне, протогосударства обучали и оплачивали деятельность чиновников, занятых оценкой стоимости земли и имущества для целей налогообложения. Эти чиновники отвечали за то, чтобы землевладельцы платили соответствующие налоги в зависимости от стоимости и продуктивности их владений аналогично тому, как это было организовано в Египте. Знаменитый \href{https://en.wikipedia.org/wiki/Code_of_Hammurabi}{Кодекс} вавилонского царя Хаммурапи содержит множество сведений об институте прав собственности и принципах налогообложения, что свидетельствует об использовании оценки в качестве инструмента государственного управления~\cite{NemetNejat1998}.

\paragraph{Демократические и республиканские полисы}

\subparagraph{Эллинистические полисы}
В Древней Греции, особенно в Афинах, оценка имущества использовалась для различных административных целей, в том числе для налогообложения и несения военной службы. Для оценки стоимости имущества гражданина использовалось понятие ``tim\={e}ma'' ($\tau\iota\mu\eta\mu\alpha$). Это достаточно широкий институт организации жизни греческого полиса. Данная система включала в себя оценку имущества, которая использовалась для различных административных и гражданских целей. Эта система играла важную роль в определении финансовых и социальных обязанностей человека в полисе и, в определённой мере, являлась собой частью общей системы ``социального рейтинга'' жителя полиса. Как известно, изначально демократия предполагала участие в управлении делами общества только достойных граждан. Одним из критериев статуса, определяющего баланс прав и обязанностей каждого гражданина, являлась стоимость имущества, определённая в рамках института оценки tim\={e}ma.

В своей работе ``Политика'' Аристотель обсуждает различные аспекты оценки имущества и распределения общественных обязанностей в зависимости от богатства, что позволяет понять философские и практические основы системы tim\={e}ma~\cite{Jowett1885}. Система tim\={e}ma была неотъемлемой частью функционирования афинского общества, обеспечивая систематическую оценку имущества, на основе которой определялись гражданские права и обязанности. Эта древняя практика заложила основу для многих современных концепций имущественного налогообложения и распределения общественных обязанностей~\cite{Hansen1999}, \cite{Finley1983}.

\subparagraph{Римская Республика и Империя}
В Древнем Риме возникла ещё более развитая система ``census'', предполагающую высокую \href{https://www.britannica.com/science/census/Modern-census-procedure}{стандартизацию процесса оценки имущества}~\cite{Britannica:census}. Во-первых, был установлено единое правило, предполагающее переоценку один раз в пять лет, во-вторых была учреждена специальная должность ``censor'', в-третьих был стандартизирован сам процесс, предполагающий подачу сведений об имуществе в натуральной форме со стороны их собственников, актуализация списка которых также была частью процедуры, и дальнейшую денежную оценку этого имущества со стороны censor~\cite{Walbank2006}, \cite{Garnsey2015}, \cite{Livy2002}. Римский census был сложной системой, которая играла важнейшую роль в управлении римским государством, влияя на налогообложение, военную службу и социальную иерархию. Её наследие можно увидеть в современных системах кадастровой оценки имущества и переписи населения.

Следующим важным новшеством, определившим развитие системы оценки до наших дней, стала система централизованного земельного кадастра. Ключевые элементы данной системы:
\begin{description}
    \item[Кадастровая съёмка:] Римляне проводили подробные исследования с использование методов триангуляции, применяемых до сих пор, чтобы нанести на карту и зафиксировать владение землёй. Эти исследования включали в себя точное измерение земельных участков, которые фиксировались в официальных документах, известных как кадастры.
    \item[Публичность кадастрового учёта:] Кадастры велись официальными органами и использовались для разрешения споров, планирования общественных работ и управления налогообложением. Эти записи служили надёжным средством проверки прав собственности на землю и ее стоимости.
    \item[Институт  Agrimensores или Gromatici:] Кадастровая система была неотъемлемой частью римской администрации, как во времена Республики так и Империи. По мере развития системы управления Республикой, вопрос учёта и оценки земли потребовал дополнение системы цензоров. Агрименсоры иначе называемые громатики, отвечали за точное измерение и регистрацию границ земли. Кроме того, они выполняли роль судей по земельным спорам, а также выступали в роль экспертов  по вопросам определения её стоимости. Со временем они выделились в отдельное богатое и уважаемое сословие.
\end{description}

Высокая для своего времени эффективность кадастрового учёта и стандартизированной оценки дала развитие системе обеспеченного кредитования, при которой имущество использовалось в качестве залога для получения кредита. Эта система была тесно связана с оценкой и кадастровым учётом. Она существовала в двух формах:
\begin{description}
    \item[Hypotheca] В этом случае заёмщик мог заложить свою собственность в качестве обеспечения, не передавая владение кредитору. Имущество выступало в качестве залога, гарантируя, что кредитор сможет потребовать его в случае невыполнения заёмщиком обязательств по кредиту.
    \item[Pignus] Другая форма обеспеченного кредита, при которой владение имуществом передаётся кредитору до погашения долга.
\end{description}
Совокупность систем ценза, кадастрового учёта и наличие специализированного института экспертов по измерения и оценке, обеспечивала понятную сегодня систему реализации различных прав в области собственности со стороны многих участников гражданских правоотношений~\cite{Veyne1996}. Ответственными за оценку залогового имущества и его учёт лицами являлись агрименсоры и цензоры.

Следует отметить, что все вышеперечисленные системы оценки, начиная с самых древних, основывались на принципах капитализации земельной ренты. Таким образом, именно доходный подход является первым известным человечеству.

Следующей областью, требующей проведение оценки, являлась сфера таможенных платежей. Для эффективной реализации интересов городов при одновременном обеспечении справедливого учёта прав торговцев был создан отдельный институт 'portoria'. Поскольку расчёт таможенных платежей осуществлялся на базе стоимости товаров, возникла необходимость в оценке данных товаров. Таким образом, можно говорить о возникновении сравнительного подхода. В отличие от оценки более-менее единообразной земли, оценка различных товаров, не имеющих единое ценообразование, вызывала существенные затруднения со стороны aediles и иных официальных лиц. Проблемы усугублялись как активной практикой занижения стоимости товаров со стороны торговцев, так и злоупотреблениями со стороны официальных лиц. При этом, неверно воспринимать Римскую Республику, а затем и Империю в рамках наших понятий о государстве, соответствующих эпохе Модерна, представляющих государство как единую централизованную систему. Практики оценки как и всего администрирования могли сильно различать от территории к территории~\cite{Oleson2008}, \cite{Sidebotham1986}. Ответом на усложнение экономической жизни стало возникновение в провинциях института 'publicani'. Сбор таможенных пошлин часто поручался частным подрядчикам. Они отвечали за оценку стоимости товаров и взимание соответствующих пошлин. Они получали право на сбор налогов и пошлин в результате тендера и должны были предоставить государству фиксированную сумму, а излишки оставляли себе в качестве прибыли. При этом, вопрос оценки объектов налогообложения и пошлин переходил в частные руки в область андеррайтинга. При этом за колониальными чиновниками оставался общий надзор за их деятельностью. Таким образом, можно сказать о том, что именно институт publicani следует считать первым опытом проведения оценки частными структурами, а не магистратами либо иными общественными институтами полисов. При этом, с учётом деятельности publicani в целом, можно сделать однозначный вывод, что первая частная оценка носила сугубо внутренний характер, являясь вспомогательной деятельностью publicani, являясь частью системы андеррайтинга~\cite{Weaver2008}.

Ещё одним направлением оценки в Риме являлась судебная экспертиза~\cite{Smits2002}. Praetors либо иные магистраты (в зависимости от территории) осуществляли правосудие и отвечали в том числе за справедливое рассмотрение споров имущественного характера. Именно в рамках римской судебной системы возник институт судебных экспертов 'peritus (мн. ч. periti)', т.\,е. лиц, обладающих специальными познаниями в той или иной области, необходимыми для принятия законного и обоснованного решения со стороны магистрата~\cite{Mommsen1985}. В вопросах определения стоимости в качестве periti чаще всего выступали агрименсоры, инженеры и архитекторы. При этом, обязанность выступать в роль peritus по требования суда являлась гражданским долгом и частью профессии~\cite{DariMattiacci2020}, \cite{Mousourakis2003}. Римское право было настолько развито, что сложно перечислить все случаи, в которых могла требоваться оценка. В качестве примера можно привести разработанные римскими юристами принципы Negotiorum gestio --- ситуации, когда gestor --- лицо предпринявшее меры для сохранения чужого имущества в опасной ситуации, вправе требовать обоснованное возмещение понесённых им расходов со стороны владельца данного имущества --- dominus negotii. Вопрос размера таких расходов мог являться предметом спора, требующего проведение оценки.

Следующей важной областью, создающей спрос на оценку, был финансовый сектор. Существовало четыре основных вида банковских институтов, каждый из которых выполнял свои уникальные функции и действовал в своё правовом поле.
\begin{description}
	\item[Argentarii] Деятельность данных специалистов в наибольшей мере напоминает деятельность современных коммерческих банков. Они принимали депозиты и осуществляли кредитование, в т.\,ч. с обеспечением залогом. В случае залога недвижимости,оценка её стоимости проводилась, как было сказано выше, с помощью агрименсоров и цензоров. Оценка иных видов имущества заёмщиков проводилась самостоятельно либо с привлечением Coactores, речь о которых пойдёт ниже~\cite{Andreau1999}, \cite{Billeter1939}.
	\item[Mensarii] Государственные банкиры, чья деятельность содержала элементы обязанностей современных центральных банков, казначейства и министерства финансов. В их обязанности входили в т.ч. поддержание ликвидности, государственные займы равно и государственное кредитование важных сфер, финансирование некоторых государственных расходов, в т.ч. для поддержания продовольственной безопасности, контроль за расходованием предоставляемых средств и т.\,п. задачи~\cite{Wild_1977}. Задачи оценки стоимости, стоящие перед ними, носили, как правило контрольный характер. В частности, они могли проверять соответствие цен товаров, закупаемых с привлечением выдаваемых ими средств. Оценка не была важной частью деятельности mensarii и выполнялась ими самостоятельно~\cite{Rostovtzeff1957}.
	\item[Coactores] Специалисты по взысканию долгов и продаже имущества на аукционах. Часто также занимались микрокредитованием. Практиковали самостоятельную оценку различного движимого имущества, т.\,ч. товаров в обороте. Часто работали совместно с Argentarii в их интересах~\cite{Crook1994}~\cite{Hollander2007}.
	\item[Nummulari] Специалисты по денежному обращению и обмену монет. В их обязанности входили размен монеты, обмен монет, выпущенных на различных территориях, экспертиза и анализ состава металла или сплава. Специализировались на оценке металлов, в т.\,ч. экспертизе подлинности состава~\cite{Howgego1997}, \cite{Sutherland1974}.
\end{description}
Помимо банковской сферы, в Древнем Риме существовали институты, не являющиеся в подлинном современном смысле страхованием, однако предвосхитившие его возникновение.
\begin{description}
	\item[Fenus Nauticum] Система займов для морской коммерции, в которой возврат инвестиции с процентами в пользу инвестора осуществлялся только в случае успешного безаварийного завершения торгового плавания и сохранения груза до порта назначения. Для определения размера безопасного займа и требуемой доходности обе стороны нуждались в актуарных расчётах, включающих в т\,ч. оценку стоимости груза и корабля. Ставки по таким займам были выше, чем по займам тем же заёмщикам для других целей, поскольку включали премию за риск вероятностного наступления события. Такие сложные механизмы стали возможны благодаря удивительно развитой правовой системе Рима, разработавшей идею возможности событий вероятностной природы и соответствующие ей правовые нормы~\cite{Wild1977}, \cite{Andreau1999}.
	\item[Общая авария (General Average)] Данное понятие было разработано римскими юристами и стейкхолдерами морской торговли задолго до известной нам его версии \href{https://www.investopedia.com/terms/y/york-antwerp-rules.asp}{York Antwerp Rules 1890 года}~\cite{Investopedia1890}. Согласно данному правилу, в случае необходимости, экипаж может пожертвовать частью груза и (или) судового оборудования в целях спасения корабля и остального груза, а также жизней экипажа и пассажиров. При этом, убытки от данных действий несёт не только непосредственный собственник погибшего имущества, но все стейкхолдеры плавания: собственник корабля (судна), владельцы всех грузов на борту и т\,д. пропорционально своему участия в плавании. Естественно, данная практика требовала проведение оценки стоимости как корабля и его отдельных элементов, так и каждого элемента груза~\cite{Casson2020}, \cite{Bagnall2004}.
	\item[Общества взаимной помощи] Римские коллегии --- предтечи средневековых цехов, являющихся, в свою очередь, прообразом профессиональных сообществ, в т.\,ч. и оценщиков, практиковали сбор небольших сумм со своих членов в общий фонд, их которого осуществлялись выплаты в их пользу в случае пожара, кражи или иных негативных событий. Очевидно, что оценка имущества была необходима как для определения рациональной размеры взносов, так и для определения справедливой компенсации в случае наступления неблагоприятного события~\cite{Smith1998}, \cite{Waltzing1895}.
\end{description}

В целом, можно сказать, что уже в римский период сформировалась целая система разнообразной внутренней оценки, осуществляемой как магистратами, так и частным бизнесов в рамках своей деятельности.
\paragraph{Средние века}
С учётом утраты связности и формирования локализованных очагов экономической активности, а также общего падения уровня развития институтов равно как и относительно низкой информационной обеспеченности, сложно говорить о какой-то единой системе оценки в данный период. Конкретные практики могли отличаться от города к городу и от феода к феоду, однако имеющаяся информация указывает на то, что институты внутренней оценки существовали на протяжении всего периода Средневековья. Основными заинтересованными в оценке структурами и лицами были:
\begin{itemize}
	\item Феодалы-землевладельцы;
	\item Города, а также существовавшие в них профессиональные гильдии (цеха);
	\item Церковь.
\end{itemize}
Далее, будет приведён краткий разбор практики оценке в Средние Века с преимущественной опорой на практики Англии. Современному человеку, выросшему и получившему образование в парадигме национальных государств эпохи Модерна, может показаться удивительным сосуществование, как минимум трёх форм организации людей, которые могли существовать
\subparagraph{Феодальная оценка}
Основным активом лордов, часто определяющим их положение во всей системе феодальной иерархии, была земля, распределяемая от вышестоящих к нижестоящим. Ценность земли определялась, в первую очередь, её продуктивностью, и, в какой-то мере, местоположением. Оценка могла носить достаточно субъективный характер, поскольку вся система феодальных отношений и передача активов нижестоящим вассалам в обмен на их службу своего лорду, существенно отличалась от привычных нам и, ранее существовавших в классический, период рыночных практик. Тем не менее, экономика сильнее любых политических систем, вследствие чего определённые системы оценки существовали и внутри феодов.

Одним из основных институтов, осуществлявшим регулирование различных сфер жизни и требовавшим проведение оценки, был манориальный суд, осуществляемый лордом поместья и имевший силу только в его пределах, а также в отношении его жителей. Каждое поместье имело собственный свод правил и записей о правах на землю и существующих договорах и обязательствах, называемый 'custumal'. Такое объединение регулирования абсолютно всех законов, конкретных договоров и инвентарных записей в одном документе являлось существенной примитивизацией относительно практики римского периода. Однако даже такой упрощённый подход требовал проведение хотя бы простейших кадастровых работ и оценку средней продуктивности земли. Это было необходимо для определения размера арендной платы и иных обязательств, возникающих из землепользования.

Сами манориальные суды, уступая во всех отношениях римским судам, тем не менее были прогрессивным институтом по сравнению с существовавшим до него \href{https://en.wikipedia.org/wiki/Trial_by_ordeal}{``судом испытанием''}~\cite{trial_ordeal} и \href{https://en.wikipedia.org/wiki/Compurgation}{``судом клятвой''}~\cite{compurgation}. Манориальные суды основывали свои решения на custumal, состояли из 12 жюри под председательством лорда поместья и, чаще всего, разбирали дела, так или иначе, имеющие отношение к вопросам имущества и учёта его стоимости: размер арендной платы, возмещение ущерба, раздел наследства, взыскание долгов, справедливое вознаграждение за службу своему лорду и т.\,п. Оценка стоимости, при необходимости, осуществлялась судом самостоятельно и не имела строгие формализованные рамки~\cite{Maitland1907}.

Одним из наиболее известных сохранившихся до наших дней памятников средневековой оценки является \href{http://www.domesdaybook.co.uk/index.html}{``Книга Судного Дня (Domesday Book)''}~\cite{domesdaybook}, содержащая достаточно детальную перепись имений на территории Англии, включающую в себя классификацию земель, а также их оценку по состоянию на три даты: момент смерти Эдуарда Исповедника, момент возникновения права на землю у текущего правообладателя, момент составления переписи. Помимо описания самой земли, перепись содержала сведения о существующих доходных объектах, таких как мельница, солеварни т.\,п. Кроме того, описания участков включали сведения о потенциально возможных улучшениях и предполагаемой стоимости после их проведения. Стоимость земли рассматривалась с точки зрения приносимого ей годового дохода.

Ещё одним важным институтом, требующим проведение оценочных процедур, являлась Система Открытых Полей, имевшая место в Англии параллельно с кельтскими, датскими и романскими практиками. Данная система предполагала существование единой пашни и пастбищ, совместно обрабатываемых жителями манора. При этом, земля лорда также находилась среди земель крестьян. Размер положенного надела определялся, исходя из продуктивности земли, и измерялся в интересных единицах --- гайдах, являвшихся одновременно единицей площади и общей продуктивности и имевших разные значения в зависимости от средней продуктивности земли в различных графствах. Обработка таких полей велась совместными усилиями равно и выпас скота осуществлялся на равных условиях. При этом, распределение урожая зависело от площади полосы земли конкретной семьи, её участия в обработке своим инвентарём и животными и иных факторов. Всё это требовало проведение оценки~\cite{Powell2011}, \cite{McCloskey1972}.

\subparagraph{Городская и цеховая оценка}
Средневековые города существовали достаточно автономно, не входили в феодальную систему и в какой-то мере являлись аналогом античных полисов. С учётом локального характера жизни городов и отсутствия единого подхода к управлению процессами в них, описание процесса оценки в них возможно лишь в первом приближении. Как и в случае с феодальной оценкой, дальнейшее описание будет основываться в основном на практиках средневековой Англии. Оценка в городах существовала в двух форматах:
\begin{itemize}
    \item оценка в рамках деятельности профессиональных гильдий (цехов);
    \item оценка в публичных интересах города.
\end{itemize}
В первом случае, оценка являлась частью системы товарной экспертизы, разработки мер и весов, а также служила для формирования цен на продукцию цеха с учётом её себестоимости и требуемой прибыли. С учётом того, что цены определялись не индивидуально, но решением всего цеха, в определённой мере можно сказать, что это одно из первых доказанных свидетельств применения затратного подхода.

Во втором случае, речь шла в основном о налоговой и таможенной оценке. Также в ряде случаев оценка требовалась в целях регулирования цен на отдельных рынках. Естественно, существовала и судебная оценка. Во всех случаях, её осуществление проводилось силами самих городских магистратов. Интересным фактом является то, что вопросы оценки являлись частью системы конфедерационных отношений между городами. Так например \href{https://www.britannica.com/topic/Cinque-Ports}{Союз Пяти Портов (Cinque Ports)}~\cite{cinqueports}, среди прочего, рассматривал вопросы следующих институтов англо-саксонского и англо-норманнского права:
\begin{description}
    \item[Toll and Team] Право лорда либо города на взымание платы за завоз на его территорию товаров либо их перемещение через неё. Поскольку в случае Союза Пяти Портов (на самом деле количество городов членов конфедерации доходило до 40) речь шла о единой таможенной территории, существовала необходимость системы единого учёта, определения стоимости, а главное справедливого распределения собранной платы между участниками конфедерации. Это требовало единого подхода к вопросу оценки как самих товаров так и вклада каждого города в экономику конфедерации.
    \item[Waif and Stray] Право на обращение бесхозной вещи либо потерявшегося животного в собственность владельца территории, на которой она была обнаружена. При этом, такое обращение было возможно только по прошествии определённого периода времени. Поскольку вопрос первичного обнаружения вещи, в особенности животных способных перемещаться между участками конкретных владельцев, мог носить спорный характер, для его мирного решения практиковался в т.\,ч. раздел стоимости вещи между несколькими землевладельцами. Что естественно требовало проведение оценки согласно единым правилам Конфедерации. При этом, в случае обнаружения владельца вещи до момента обращения вещи в собственность нашедшего, владелец мог истребовать вещь обратно, возместив нашедшему расходы по содержанию данной вещи. В ряде случаев, это также могло потребовать проведение оценки в ходе судебного спора.
    \item[Flotsam, Jetsam, Lagan and Derelict] Данная обширная тема морского права была связана как с вопросами компенсации за участие в спасении гибнущего корабля и груза на нём так и с вопросами прав собственности на потерянный или выброшенный груз, а также остатки корабля. Данные вопросы практически всегда так или иначе требовали проведение оценки. Существовало много нюансов, определяющих дальнейших ход разбирательства. Например, судьба права зависела от того, упал ли груз случайно или был выброшен, находится ли он в плавучем либо прогружённом состоянии, известно ли точное местонахождение затонувшего груза, а также ряда иных обстоятельств. Кроме того, могли возникать споры касательно принадлежности груза, найденного в прибойной зоне. Также стороны могли сами неверно определить правовой статус, в результате чего речь могла идти лишь о постфактум восстановлении нарушенного права путём выплаты компенсации.
\end{description}
В целом можно сказать, что оценка, осуществляемая в рамках обычной коммерческой деятельности, а также судебного характера имела широкое распространение и могла носить характер межгородского института, работающего по единым правилам. Так, в случае Союза Пяти Портов, данные правила разрабатывались и применялись конфедеративным органами управления такими как The Court of Shepway и Brodhull~\cite{Pirenne2014}, \cite{Epstein2001}.
\subparagraph{Церковная оценка}
Средневековая церковь представляла собой институт, занимающий доминирующее положение во всей экономике. Особенно высокая и часто исключительная роль в области финансовой деятельности позволяла не только успешно вести собственную хозяйственную деятельность, но и в определённой мере регулировать экономические отношения всех прочих субъектов. Церковная организация в формате сетевой структуры позволяла обеспечивать движение капитала, выступать гарантом в отношениях между субъектами, управлять ликвидностью и ценами в целых регионах. К этому следует добавить, что церковь также была крупнейшим землевладельцем и крупнейшим получателем налогов. Кроме того, она обладала собственной обширной судебной юрисдикцией. При этом, именно церковь в наибольшей мере оставалась носителем римских правовых знаний. В определённой мере можно сказать, что практика церковной оценки включала в себя обобщение всех существовавших практик и представляла собой наиболее развитую систему оценки. Далее будет приведён краткий перечень деятельности церкви в сфере оценки~\cite{Southern1985}, \cite{Moore2008}.

Прежде всего, следует сказать, что именно церковь являлась крупнейшим землевладельцем. При этом, непосредственная операционная деятельность могла осуществляться как в форме передачи земли арендаторам с получением от них арендной платы, так и в форме самостоятельного управления с использованием наёмного либо иного труда. Помимо самого сельскохозяйственного производства, церковь осуществляла деятельность в области логистики и торгового посредничества, в частности посредством закупки сельскохозяйственной продукции, её централизованного хранения и дальнейшей продажи в удобный момент с учётом рыночной конъюнктуры. Кроме того, хорошо защищённые церковные хранилища могли использоваться другими участниками  Всё это требовало проведение оценки как текущей так и предполагаемой стоимости активов и продукции~\cite{Harvey1996}, \cite{Haverkamp1996}.

Влияние церкви было настолько значительным, что она не только получала доходы от принадлежащих ей активов, но и собирала налоги со светской экономики. Среди наиболее известных налогов можно назвать церковную десятину, собираемую в пользу локальных церковных структур, и ``Денарий Святого Петра'', собираемый в пользу Ватикана. Расчёт налоговой базы для десятины осуществлялся на достаточно сложной основе, требующей проведение ежегодной оценки. В частности, в Англии расчёт такого налога мог осуществляться на основе усредняемых за семь предшествующих лет показателей дохода и стоимости имущества. Такой подход сам по себе показывает достаточно высокий уровень понимания природы стоимости, цикличности экономики и необходимости усреднения результата деятельности для учёта баланса интересов~\cite{Lynch2014}, \cite{Bolton1983}.

Ещё одним важным направлением деятельности церкви, требующим проведение оценки, являлось осуществление правосудия. Низовые церковные суды в целом были схожи с манориальными. Они также имели юрисдикцию только в отношении своих земель и их жителей. Важной особенностью церковных судов, равно как и прочих органов церкви являлась крайне подробная запись всех обстоятельств дел, в т.\,ч. описания имущества и его стоимости. При этом, следует отметить, что предметом рассмотрения таких судов являлись в т.\,ч. и вопросы налогов и иных платежей в пользу церкви. Несмотря на очевидны конфликт интересов, такие суды всё же имели место и доводы стороны, оспаривающей размер налога, аренды либо иного платежа рассматривались по существу в т.\,ч. с учётом именно оценочной экспертизы~\cite{Brundage1999}, \cite{Hartmann2008}.

Рассмотрение вопросов церковной оценки было неполным без анализа такого важнейшего аспекта деятельности церкви как банкинг. Средневековая церковь действительно сыграла значительную роль в развитии ранних банковских систем, используя свою обширную сеть монастырей, соборов и церковных учреждений по всей Европе. Эта протобанковская деятельность включала в себя такие функции, как приём депозитов, кредитование и перевод средств между регионами. Оценка активов, доходов и обязательств занимала важное место в этой банковской деятельности, обеспечивая финансовую стабильность церкви и её способность предоставлять кредиты.

\begin{description}
    \item[Деятельность при приёму вкладов] Церковь, особенно монастыри и крупные церковные учреждения, часто служили хранилищами для мирян и дворян. Эти вклады могли представлять собой как собственно деньги, так и движимое имущество и даже землю. Если в первом случае всё достаточно очевидно, то приём имущества во вклад требовал проведение достаточно серьёзной оценки, требующей учёт будущих денежных потоков от его использования в период временного операционного управления со стороны церковного института. Необходимость возврата имущества и части дохода от его использования предполагала высокую ответственность со стороны финансовых специалистов церкви в части корректного определения стоимости.
    \item[Кредитование] Церковь ссужала деньги дворянам, купцам, городам, цехам, также  другим религиозным учреждениям. Эти займы часто выдавались под залог  земли или будущих доходов, в т.\,ч. налогов, включая саму церковную десятину. Это требовало весьма тщательную оценку, гарантирующую защиту интересов церкви. Проценты по этим займам иногда были замаскированы под сборы или другие платежи в связи с ограничением ростовщичества. В то же время, некоторые церковные институты открыто вели деятельность по выдаче займов.
\end{description}

Ещё одним важным направлением банковской деятельности церкви было осуществлением трансфера средств. Опасности физического перемещения ценностей заставляли многих обращаться к церкви. Церковные учреждения выпускали аккредитивы, позволявшие купцам и иным путешественникам получать доступ к своим средствам в отдалённых местах без собственно наличия этих средств при себе. Это существенно снижало риски и способствовало развитию торговли. Естественно, такая услуга со стороны церкви носила платный характер и требовала проведение оценки залога, предоставляемого клиентом в обеспечение аккредитива.

Папский двор в Риме выступал в качестве центрального банковского органа, управляя крупными переводами средств по всей Европе, а также предоставляя ликвидность нижестоящим церковным учреждениям. Для обеспечения данной и иных функций, Ватикан собирал собственный глобальный налог, именуемый ``Денарий Святого Петра''. Являясь крупнейшим держателем долговых обязательств, Святой Престол часто был вынужден выступать арбитром во многих спорах. Помимо этого, в случае очевидных проблем с получением вложенных средств, практиковались по сути коллекторские операции, маскируемые под ``крестовые походы'' и походы против ``еретиков''. При этом, реализация таких мероприятий, как правило, осуществлялась посредством привлечения иных должников в обмен на списание их долгов, маскируемое под ``отпущение грехов''. Тождественность понятий ``долг'' и ``грех'' в Средние Века подтверждается рядом фактов, включая альтернативные версии некоторых текстов. Таким образом, администрации Ватикана приходилось решать достаточно сложные задачи, включающие расчёт по сути инвестиционной стоимости того или иного проекта, требующего вовлечение значительных ресурсов и списания ряда активов в целях получения преимуществ от получения иных долгов и обращения взыскания на имущество должников~\cite{Hollister2001}, \cite{Lynch2014}.

Для обеспечения вышеперечисленных функций, церковью были разработаны некоторые достаточно инновационные для своего времени инструменты вроде проведения оценки не только конкретных залогов, но и учёта стоимости и рисков по всему портфелю в целом, планирования денежных потоков с учётом фактора риска, прообраза двойной записи в системах учёта, получившей распространение в последующие эпохи и существующей до настоящего времени. С учётом того, что многие операции носили длящийся характер, регулярная переоценка активов и обязательств с учётом изменения обстоятельств была частью обычной деятельности церковных учреждений, в особенности крупных~\cite{Neal1999}, \cite{Postan1972}.

\paragraph{Ренессанс}
Эпоха Возрождения (XIV-XVII века) ознаменовалась глубокими изменениями в европейском обществе, вызванными культурным возрождением, Великими географическими открытиями и ранними стадиями капитализма. Практика оценки в этот период претерпела определённые изменения, отражая растущую сложность экономической деятельности, расширение торговли и появление финансовых институтов. В некотором смысле, можно сказать, что в этот период возросла роль инвестиционной оценки.

Одним из важных обстоятельств, определивших дальнейшее развитие экономики и оценки, стали Великие Географические Открытия. Открытие новых земель и морских путей в Америку, Азию и Африку открыло беспрецедентные возможности для торговли. Оценка товаров, кораблей и грузов становилась все более сложной, поскольку европейские купцы имели дело с такими экзотическими товарами, как пряности или шёлк, которые требовали точной оценки для определения их стоимости на европейских рынках. Помимо этого, возник существенный приток драгоценных металлов, что в корне изменило финансовую систему Европы. Вместо стабильных цен, существовавших на протяжении жизней нескольких поколений и колебавшихся вокруг примерно одних значений в зависимости от природных условий, возникла динамическая система ценообразования, в которой сами эталоны стоимости золото и серебро имели переменные цены. Помимо этого, нередко возникала ситуация недостаточности капитала у самих операторов морских экспедиций. Это привело к развитию института финансирования плаваний со стороны различных частных лиц и институтов, включая феодалов, банкиров и церкви. Подобные инвестиционные проекты, с одной стороны, являлись высокорисковыми, с другой, могли иметь рентабельность до 3000~\%. Участие в таких проектах требовало проведение оценки их инвестиционной стоимости со стороны как финансовых специалистов инвесторов так и непосредственных исполнителей~\cite{Braudel1983}, \cite{Pomeranz2021}.

Вторым обстоятельством, навсегда изменившим мировую экономику и как следствие оценку, стало возникновение акционерных обществ и в более широком смысле юридических лиц как таковых. Революционность идеи заключалась в разделение собственности и управления, а также возможности формализации отношений между субъектами на основе договоров безотносительно личностей участников~\cite{Lopez2001}, \cite{Luzzatto1954}. Прообразом таких обществ стали итальянские \textit{``Commenda''}, возникшие на излёте Средних Веков. Данные виды партнёрства предполагали участие, как правило двух сторон. Одна из которых, именуемая инвестировала капитал в некоторое коммерческое мероприятие, носившее разовый либо лимитированный характер,






в юрисдикциях, где формализованный институт оценки существует   с давней традицие Деятельность обеих категорий направлена на установление того или иного вида стоимости. При этом, между ними существует сущест
\subsection{Задача}

\subsection{Функция}

\subsection{Знания}

\subsection{Ценообразование}

\section{Результаты}\label{sec:results}

\section{Проблемы экзамена}

\section{Существующие предложения}

\section{Благодарности}\label{sec:acknowledgements}
Автор выражает благодарность П.\,А.~Козину, В.\,Н.~Мягкову, А.\,А.~Слуцкому, В.\,А.~Шогину за консультации и орфографические правки при подготовке данного материала.





\printbibliography

\end{document}
